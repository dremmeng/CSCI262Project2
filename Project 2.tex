% LaTeX Article Template - customizing page format
%
% LaTeX document uses 10-point fonts by default.  To use
% 11-point or 12-point fonts, use \documentclass[11pt]{article}
% or \documentclass[12pt]{article}.
\documentclass{article}

% Set left margin - The default is 1 inch, so the following 
% command sets a 1.25-inch left margin.
\setlength{\oddsidemargin}{0.25in}

% Set width of the text - What is left will be the right margin.
% In this case, right margin is 8.5in - 1.25in - 6in = 1.25in.
\setlength{\textwidth}{6in}

% Set top margin - The default is 1 inch, so the following 
% command sets a 0.75-inch top margin.
\setlength{\topmargin}{-0.25in}

% Set height of the text - What is left will be the bottom margin.
% In this case, bottom margin is 11in - 0.75in - 9.5in = 0.75in
\setlength{\textheight}{8in}
\usepackage{fancyhdr}
\usepackage{float}
\usepackage{mathtools}
\usepackage{amsmath}
\usepackage{amssymb}
\usepackage{graphicx}
\graphicspath{ {./} }

\setlength{\parskip}{5pt} 
\pagestyle{fancyplain}
% Set the beginning of a LaTeX document
\begin{document}

\lhead{Drew Remmenga CSCI 262}
\rhead{Project \#2}
%\lhead{Independent Study}
%\rhead{R Lab}

First we find every single permutation of the boards. Then we prune off all the boards where the layout is impossible or that 'x' doesn't go first. Then we count up all the victories and tally them for display. \par
\par
The program compiles in 1.5 seconds. This is because we evaluate every permutation even ones with all 'x's and all 'o's before we start pruning them off. The output is correct since we start with the very first permutation being the first sorted and gradually evaluate every other permutation. \par
\par
If we didn't evaluate every permutation the program would be much faster. If we could exploit symmetry to solve this problem I believe the program would be much faster as well. It would shave off time for both the pruning and the evaluation. 

\end{document}
